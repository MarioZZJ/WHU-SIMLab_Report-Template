\documentclass[12px]{article}
\usepackage{indentfirst}
\setlength{\parindent}{2em}
\usepackage{ctex}
\usepackage{makecell}
\usepackage[top=2cm ,bottom=2cm ,left=2cm ,right=3cm]{geometry}
\usepackage{amssymb}
\usepackage{fancybox}
\begin{document}
    \begin{center}
        \large{实验报告} \\
        \begin{tabular}{|m{1.98cm}<{\centering}|m{3.25cm}<{\centering}|m{1.75cm}<{\centering}|m{3.27cm}<{\centering}|m{2.33cm}<{\centering}|m{2.35cm}<{\centering}m{0cm}|} \hline
             姓名 &    & 学号 &  & 年级 &  &\rule{0cm}{1.4cm} \\ \hline
            \makecell[c]{成绩 \\ \footnotesize{(考核档次或分数)}} &   & 指导老师 &   & 专业 &   &\\ \hline
            \multicolumn{2}{|l}{\makecell[l]{实验类型:\\ $\blacksquare$ 独立实验课\\ $\square$ 含实验的理论课\\ $\square$ 计划外自选开放实验\\ $\square$ 学生自主式开放实验\\ $\square$ 大学生科技竞赛}}
                & \multicolumn{2}{|l}{\makecell[l]{实验日期:\\ 2020-2021学年第2学期 \\ 第\ 6 \ 周}}
                & \multicolumn{2}{|l}{\makecell[l]{实验学时数:\\ \\ 3 \ 学时 }} &\\ \hline
            \multicolumn{6}{|l}{相关课程: } &\rule{0cm}{1.12cm} \\ \hline
            \multicolumn{6}{|l}{相关科研项目:无} &\rule{0cm}{1.12cm} \\ \hline
            \multicolumn{2}{|c|}{实验名称} & \multicolumn{4}{c}{ } &\\ \hline
            \multicolumn{6}{|p{14.92cm}}{
                \makecell[l]{
                \textbf{一、预习部分(实验目的、实验基本原理等)} \\
                    实验原理:\\
                    \\
                    \\
                    \\
                    主要仪器设备(含必要的元器件、工具):\\
                    \\
                    \\
                    实验目的:\\
                    \\
                    \\
                    \\
                    \\
                    实验内容:\\
                    \\
                    \\
                    \\
                    \\
                    \\
                }
            } &\\ \hline

        \end{tabular}
    \end{center}


    \newpage
    \fancypage{\fbox}{}
    \textbf{二、实验操作部分(可续页)}

        1. 实验数据、表格及数据处理

        2. 实验操作过程(可用图表示)

        3. 结论

        startUrProcess


    \newpage
    \fancypage{\fbox}{}
    \textbf{三、实验效果分析(包括仪器设备等使用效果、实验完成情况)}

    startUrSummary
    \newpage
    \fancypage{}{}
    \begin{center}
        \begin{tabular}{|m{1em}|m{15.5cm}|}
            \hline
            教师评语 & \rule{0em}{21cm} \\
            \hline
        \end{tabular}
    \end{center}
\end{document}
